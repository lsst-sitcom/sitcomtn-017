\documentclass[SE,authoryear,toc]{lsstdoc}
% lsstdoc documentation: https://lsst-texmf.lsst.io/lsstdoc.html
\input{meta}

% Package imports go here.

% Local commands go here.

%If you want glossaries
%\input{aglossary.tex}
%\makeglossaries

\title{SIT-Com Image Quality Working Group Description}

% Optional subtitle
% \setDocSubtitle{A subtitle}

\author{%
Christopher W. Stubbs
}

\setDocRef{SITCOMTN-017}
\setDocUpstreamLocation{\url{https://github.com/lsst-sitcom/sitcomtn-017}}

\date{\vcsDate}

% Optional: name of the document's curator
% \setDocCurator{Christopher Stubbs}

\setDocAbstract{%
A description of the deliverables and timelines for the Image Quality Group
}

% Change history defined here.
% Order: oldest first.
% Fields: VERSION, DATE, DESCRIPTION, OWNER NAME.
% See LPM-51 for version number policy.
\setDocChangeRecord{%
  \addtohist{1}{2021-08-21}{Unreleased version from C. Stubbs}{Patrick Ingraham}
}


\begin{document}

% Create the title page.
\maketitle
% Frequently for a technote we do not want a title page  uncomment this to remove the title page and changelog.
% use \mkshorttitle to remove the extra pages

% ADD CONTENT HERE
% You can also use the \input command to include several content files.

\section{Context and Motivation}

The effectiveness of the Rubin Observatory's image stream depends strongly on image quality.
Both the size and shape of the images formed on the focal plane are factors that influence the system's scientific impact.
Achieving and sustaining the best possible image quality is therefore essential to having the project reach its full potential.

This technical note outlines the goals, structure, and scope of the effort to characterize, quantify, ameliorate, and monitor sources of PSF degradation for the Rubin observatory systems.
The 3.5 degree field of view of the camera on the Rubin telescope precludes using adaptive optics to compensate for wavefront errors introduced by the upper atmosphere.
We therefore expect that atmospheric seeing will provide a fundamental limitation to the best PSF's the system will be able to achieve.
This effort will strive to achieve the goal of seeing-limited PSF's.

\section{Scope and Deliverables}

This effort will extend from the construction phase of the Rubin observatory through commissioning and into operations.
The framework for this process entails producing a budget that quantifies the various contributions to image degradations, including:
\begin{itemize}
\item upper atmosphere seeing,
\item ground layer seeing,
\item mirror and dome seeing,
\item residual aberrations in the optical system, arising from misalignments and figure errors,
\item both low-angle scattering and undesired reflections off optical elements,
\item tracking errors and oscillations in the telescope control system, and
\item detector and signal chain artifacts.
\end{itemize}

The team will perform measurements and analyses to track these various contributions to the PSF of the Rubin telescopes.
The early availability of the Auxiliary Telescope, equipped with a detector and readout electronics identical to those used in the main Rubin camera, provides an opportunity to develop and refine methods for constructing and monitoring the image quality budget.

The scope of this effort includes obtaining and exploiting additional sources of information to tease apart the various contributions to image quality deterioration.
An important data stream will be delivered by Differential Image Motion Monitors (DIMMs), temperature metrology, and other diagnostic instruments.

Specific deliverables include:
\begin{enumerate}
\item Developing and updating an image quality budget for the Auxiliary Telescope, with associated data collection and analysis methods
\item Deploying seeing monitoring instrumentation in the Auxiliary Telescope,
\item Developing, documenting, and archiving analysis tools for image quality assessments,
\item Migrating the methods, tools, and personnel of the Image Quality Group to the main Rubin telescope once it enters the system integration phase,
\item Developing and updating an image quality budget for the Rubin Telescope, with associated data collection and analysis methods,
\item Deploying seeing monitoring instrumentation in the Rubin Telescope,
\item Ensuring that tools and methods are fully documented, and carry forward though the integration, commissioning, and operational phases,
\item Coordinating efforts from DESC and other non-project scientists and engineers, in the domain of image quality work.
\end{enumerate}

\section{Timeline and Management Stucture}

The group's initial focus will be on determining the image quality budget for the Auxiliary telescope.
This will contribute to improving the data quality from Aux Tel, but will also provide an opportunity to have a toolkit of validated methods at the ready once the main Rubin telescope begins early data collection.
While predicting dates is a challenge given the uncertainties surrounding COVID, the work on Aux Tel will likely occupy mid-calendar-2021 through some time in 2022 when SITCOM activities ramp up on the main telescope.

A target date for obtaining an initial Aux Tel image quality budget is Dec 2021, with the installation of additional in-dome instrumentation in Jan 2022.

The Image Quality Group's work will be coordinated and guided by the group Chair.
That individual will report to the head of the Rubin SITCOM effort during the commissioning phase, and to the Observatory Chief Engineer during the operations phase of the project.
The group Chair or their designate will coordinate the image quality work undertaken by external participants and members of science collaborations.

\section{Out-of-Scope Aspects}

The image quality group bears responsibility for identifying and ranking sources of PSF degradation, and recommending steps the project can take to address them.
The team does is not responsible for carrying out that work.
These recommendations can include fixes, upgrades, operational changes, monitoring and visualization tools, and additional instrumentation.

\section{Participants}

\begin{itemize}
\item Chuck Claver, Rubin System Scientist
\item Erik Dennihy, Rubin Commissioning Scientist
\item Brodi Elwood, Harvard
\item Merlin Fisher-Levine, Rubin Data Management whiz
\item Patrick Ingraham, Rubin Calibration System Engineer
\item Eske Pedersen, Harvard
\item Brian Stalder, Rubin Commissioning Scientist
\item Christopher Stubbs, Harvard
\item Tony Tyson, UC Davis
\item Elana Urbach, Harvard
\end{itemize}


\appendix
% Include all the relevant bib files.
% https://lsst-texmf.lsst.io/lsstdoc.html#bibliographies
\section{References} \label{sec:bib}
\renewcommand{\refname}{} % Suppress default Bibliography section
%\bibliography{local,lsst,lsst-dm,refs_ads,refs,books}

% Make sure lsst-texmf/bin/generateAcronyms.py is in your path
\section{Acronyms} \label{sec:acronyms}
\addtocounter{table}{-1}
\begin{longtable}{p{0.145\textwidth}p{0.8\textwidth}}\hline
\textbf{Acronym} & \textbf{Description}  \\\hline

COVID & COrona VIrus Disease \\\hline
DESC & Dark Energy Science Collaboration \\\hline
PSF & Point Spread Function \\\hline
SE & System Engineering \\\hline
SIT & System Integration, Test \\\hline
\end{longtable}

% If you want glossary uncomment below -- comment out the two lines above
%\printglossaries





\end{document}
